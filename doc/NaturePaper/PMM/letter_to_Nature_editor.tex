\documentclass{letter}
\usepackage{amsfonts,amssymb,stmaryrd,latexsym,amsmath,braket}

\begin{document}
\begin{letter}{}
\opening{Dear Nature Editor,}
We are submitting our manuscript ``Parametric Matrix Models'' for publication as an article in Nature.  This work introduces a new machine learning approach that is fundamentally different in design and strategy from other methods and achieves superior performance for scientific computing applications.  Instead of imitating the biology of neurons, parametric matrix models (PMMs) use matrix equations that emulate the physics of quantum systems.  One of the first tasks in solving any physics problem is to write down the governing equations.  While the solutions to the equations may produce highly complex phenomena, the equations themselves have a simple and logical structure.  The structure of the equations leads to important and nontrivial constraints on the properties of the solutions.  Some well-known examples include symmetries, conserved quantities, causality, and analyticity.  Unfortunately, these constraints are usually not reproduced by machine learning algorithms, leading to inefficiencies and limitations on accuracy and predictive power when applied to scientific computing applications. Furthermore, the results of many modern deep learning methods suffer from a lack of interpretability.

Instead of simply fitting the output functions using some prescribed form, PMMs learn the equations that lead to the desired outputs.  While originally designed for scientific computing, we prove that parametric matrix models are universal function approximators and show that they can be applied to general machine learning problems.  In this work, we demonstrate the performance of PMMs for a broad range of machine learning problems.  For all the challenges tested, the performance of PMMs are comparable to or exceeding that of other machine learning algorithms with the same number of parameters.  Furthermore, we demonstrate that PMMs can be integrated with neuromorphic methods to produce powerful new hybrid approaches.

For the scientific computing applications considered here, PMMs demonstrate significantly better accuracy, interpretability, and predictive power when compared with other machine learning approaches.  The performance advantage is likely due to at least two major factors.  The first is that PMMs can be designed to incorporate important mathematical structures such as operator commutation relations associated with the physical problem of interest.  The second is that PMMs produce output functions with analytical properties determined by the eigenvalues and eigenvectors of matrices.  Using the properties of sharp avoided level crossings, PMMs are able to reproduce abrupt changes in data without also producing the unwanted oscillations typically generated by other approaches. This technique of using avoided level crossings is an important part of the proof of the universal approximation theorem for PMMs.

Since all physical phenomena can ultimately be described in terms of quantum mechanics, matrix equations do exist for any observable.  Parametric matrix models take the additional step of applying the principles of model order reduction to produce accurate finite-dimensional matrix models.  While there are numerous studies that combine classical machine learning with quantum computing algorithms, PMMs use the matrix algebra of quantum mechanics as their native language.  This unusual design also sets PMMs apart from other physics-inspired and physics-informed machine learning approaches.  Rather than imposing auxiliary constraints on the output functions, important mathematical structures such as symmetries, conservation laws, and operator commutation relations can be built into the underlying matrix equations.  Once gate fidelity and qubit coherence are of sufficient quality for the operations required, PMMs can be implemented on quantum computers with techniques similar to that used in a field of quantum computing algorithms called Hamiltonian learning.

In view of the novel developments, performance advantages, and numerous applications of this work across the fields of physics, chemistry, engineering, machine learning, quantum computing, financial modeling, and applied mathematics, we believe that our article meets the very high threshold of scientific impact and broad appeal for publication in the journal Nature.

\closing{Sincerely yours,\\The Authors}
\end{letter}
\end{document}

