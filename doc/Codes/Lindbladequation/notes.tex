\documentclass[11pt,a4paper]{article}
\usepackage[utf8]{inputenc}
\usepackage[T1]{fontenc}
\usepackage{amsmath, amssymb, amsfonts, bm}
\usepackage{graphicx}
\usepackage{hyperref}
\usepackage{physics}
\usepackage{braket}
\usepackage{geometry}
\geometry{margin=2.5cm}
\usepackage{setspace}
\onehalfspacing

\title{Lindlad, Noise and possible links with PMMs}
\author{Collaboration on Noise, Open quantum systems and more}
\date{November 2}

\begin{document}
\maketitle

\section{Introduction}

This note provides an extremely abridged, and perhaps overly
practical, introduction to open quantum systems. The goal is to
provide the basis for understanding and implementing an open quantum
system using the Lindblad master equation. And then hopefully link this with PMMs.

\section{Open?}

In quantum mechanics, a system is considered \emph{closed} if it is
isolated from its environment. In contrast, an \emph{open quantum
system} interacts with its environment, which can cause the system to
lose coherence and entanglement.

All this boils down to whether or not energy is conserved. In a closed
system, energy is conserved, while in an open system, energy is not
conserved.


\section{Density Matrices}

As opposed to the wavefunction (a so-called pure state) in closed
(Hermitian) quantum mechanics, the density matrix uniquely describes
the state of a quantum system in an open system, as well as in closed
systems.

With pure states, the state of a quantum system is deterministic and
leads to probabilities of outcomes of measurements. In contrast, what
if the state itself is uncertain and therefore probabilistic? This is
what density matrices describe.\footnote{A natural question might be:
“What about probabilistic mixtures of density matrices?” Luckily, such
systems are also just described by density matrices.}

The important properties are summarized here:
\begin{itemize}
    \item The density matrix of a pure state $\ket{\psi}$ is $\rho = \ket{\psi}\bra{\psi}$.
    \item $\rho$ is Hermitian, positive semi-definite (all eigenvalues non-negative), and has unit trace:
    \[
    \rho = \rho^\dagger, \quad \lambda(\rho) \ge 0, \quad \Tr(\rho) = 1.
    \]
    \item Given a probabilistic mixture of states with probabilities $\{p_i\}$ and states $\{\ket{\psi_i}\}$,
    \[
    \rho = \sum_i p_i \ket{\psi_i}\bra{\psi_i}.
    \]
    \item For a $d \times d$ density matrix $\rho$, there exists an orthonormal basis $\{\ket{b_i}\}$ such that
    \[
    \rho = \sum_i \lambda_i \ket{b_i}\bra{b_i},
    \]
    where $\lambda_i$ are probabilities.
    \item The expectation value of an observable $X$ is
    \[
    \langle X \rangle = \Tr(\rho X) = \sum_i \lambda_i \bra{b_i} X \ket{b_i}.
    \]
    \item The time evolution in a closed system with Hamiltonian $H$ is
    \[
    \rho(t) = e^{-iHt} \rho(0) e^{iHt}.
    \]
\end{itemize}

\subsection{One Qubit Example}

The codes for these various examples are included as a separate Python code..
Consider a single qubit system that has a 50\% chance of being in
$\ket{0}$, a 25\% chance of being in $\ket{+} =
\frac{1}{\sqrt{2}}(\ket{0}+\ket{1})$, and a 25\% chance of being in
$\ket{-} = \frac{1}{\sqrt{2}}(\ket{0}-\ket{1})$. Then

\[
\rho = \frac{1}{2}\ket{0}\bra{0} + \frac{1}{4}\ket{+}\bra{+} + \frac{1}{4}\ket{-}\bra{-} =
\begin{bmatrix}
3/4 & 0 \\[4pt]
0 & 1/4
\end{bmatrix}.
\]

\subsection{Two Qubit Example}

Consider a system of two qubits that has a 90\% chance of being in the Bell state
\[
\ket{B_+} = \frac{1}{\sqrt{2}}(\ket{00} + \ket{11}).
\]
The remaining 10\% is distributed among $\ket{01}$, $\ket{10}$, and $\ket{B_-} = \frac{1}{\sqrt{2}}(\ket{00}-\ket{11})$ in the ratio 3:2:1. Then
\[
\rho = \frac{9}{10}\ket{B_+}\bra{B_+} + \frac{1}{20}\ket{01}\bra{01} + \frac{1}{30}\ket{10}\bra{10} + \frac{1}{60}\ket{B_-}\bra{B_-}.
\]

\section{Subsystems and Partial Traces}

An open quantum system is a subsystem of a closed system composed of system + environment. To describe only the system, we trace out the environment:
\[
\rho_A = \Tr_B(\rho).
\]
For a pure state $\ket{\psi} = \ket{a}\otimes\ket{b}$, the reduced density matrix is
\[
\rho_A = \sum_j (I_A \otimes \bra{j}) \rho (I_A \otimes \ket{j}).
\]

\subsection{Two Qubit Example}

From the two-qubit example,
\[
\rho_1 = \Tr_2(\rho) =
\begin{bmatrix}
61/120 & 0 \\[4pt]
0 & 59/120
\end{bmatrix},
\qquad
\rho_2 = \Tr_1(\rho) =
\begin{bmatrix}
59/120 & 0 \\[4pt]
0 & 61/120
\end{bmatrix}.
\]

\section{von Neumann Equation}

For the total density matrix $\rho_T$ of system plus environment with total Hamiltonian $H_T$,
\[
\frac{d\rho_T}{dt} = -i[H_T, \rho_T].
\]

\section{Lindblad Master Equation}

Writing $H_T = H + H_E + H_I$, where $H$ is the system Hamiltonian, $H_E$ the environment Hamiltonian, and $H_I$ their interaction, tracing out the environment and assuming weak coupling yields:
\[
\frac{d\rho}{dt} = -i[H,\rho] + \sum_k \gamma_k \left( L_k \rho L_k^\dagger - \frac{1}{2}\{L_k^\dagger L_k, \rho\} \right),
\]
where $\rho$ is the reduced density matrix, $\gamma_k \ge 0$ are decay rates, and $L_k$ are jump operators.

The Heisenberg picture form is
\[
\frac{dX}{dt} = i[H,X] + \sum_k \gamma_k \left( L_k^\dagger X L_k - \frac{1}{2}\{L_k^\dagger L_k, X\} \right),
\]
and the identity operator satisfies $\frac{dI}{dt}=0$.

\subsection{Liouvillian Superoperator}

Define the superoperator $\mathcal{L}$ by
\[
\mathcal{L}[\rho] = -i[H,\rho] + \sum_k \gamma_k \left( L_k\rho L_k^\dagger - \frac{1}{2}\{L_k^\dagger L_k, \rho\} \right),
\]
so that
\[
\frac{d\rho}{dt} = \mathcal{L}[\rho].
\]

\subsection{Vectorization: The Fock–Liouville Space}

Define $\ket{A\!\rangle\!\rangle}$ as the column-stacked vectorization of an operator $A$. The inner product is
\[
\braket{\!\braket{A|B}} = \Tr(A^\dagger B),
\]
and
\[
\frac{d}{dt}\ket{\!\braket{\rho}} = \hat{\mathcal{L}}\ket{\!\braket{\rho}},
\]
with solution
\[
\ket{\!\braket{\rho(t)}} = e^{\hat{\mathcal{L}}t} \ket{\!\braket{\rho(0)}}.
\]
Diagonalizing $\hat{\mathcal{L}}$ gives eigenvalues $\lambda_i$ and eigenvectors $\ket{\!\braket{r_i}}$ and $\bra{\!\braket{l_i}}$, allowing
\[
\ket{\!\braket{\rho(t)}} = \sum_i e^{\lambda_i t} \braket{\!\braket{l_i|\rho(0)}} \ket{\!\braket{r_i}}.
\]

\subsection{Steady State}

The steady state $\ket{\!\braket{\rho_{\text{ss}}}}$ satisfies
\[
\hat{\mathcal{L}}\ket{\!\braket{\rho_{\text{ss}}}} = 0.
\]

\subsection{Harmonic Oscillator Coupled to a Bath}

For $H = \omega a^\dagger a$ and jump operators $L_1 = \sqrt{\gamma(\tau+1)}\,a$, $L_2 = \sqrt{\gamma\tau}\,a^\dagger$, the Lindblad equation reads:
\[
\frac{d\rho}{dt} = -i[\omega a^\dagger a, \rho] + \gamma(\tau+1)\left[a\rho a^\dagger - \frac{1}{2}\{a^\dagger a,\rho\}\right]
+ \gamma\tau\left[a^\dagger\rho a - \frac{1}{2}\{aa^\dagger,\rho\}\right].
\]

See separate Python code

\section{Project 1: studies of Markovian and non-Markovian noise}

More text to come here.

\section{Project 2: roadmap for beyond zero noise extrapolation for quantum computing using PMMs}

{\bf This text will be improved upon.}


In the Noisy Intermediate Scale Quantum (NISQ) computing era, error
mitigation is vital to obtain meaningful results from quantum
algorithms running on existing hardware. This project aims to develop
a Parametric Matrix Model (PMM) which learns the underlying open
quantum system nature of today’s noisy quantum computers in order to
better understand the sources of error and directly extrapolate to
results with zero noise.



Quantum gates are often modeled idealistically as unitary operators
acting on a quantum state. However, in practice, quantum gates are
subject to noise which can be modeled as a quantum channel. The noise
can be characterized by a superoperator acting on the density matrix
of the quantum state. There are many equivalent formalisms including
the Liouville superoperator and the Kraus operator
sum~\cite{wood2015tensor}.

A simplified mathematical overview is that if a noiseless operation can be represented as a unitary operator \( U \) acting on a pure state \( |\psi\rangle \),
\[
|\psi\rangle \rightarrow U |\psi\rangle,
\]
then a noisy operation can be represented as a superoperator \( S \) acting on a vectorized density matrix \( |\rho\rangle\rangle \),
\[
|\rho\rangle\rangle \rightarrow S |\rho\rangle\rangle.
\]
Vectorization is a central operation in the Liouville formalism. The vectorization of a matrix \( A \) is formed by stacking the columns of \( A \) into a single column vector.

Any noiseless quantum circuit can be represented by a product of unitary operators \( U_1, U_2, \ldots, U_n \) acting on the initial state \( |\psi_0\rangle \),
\[
|\psi_0\rangle \rightarrow U_n U_{n-1} \cdots U_1 |\psi_0\rangle.
\]
The corresponding noisy circuit can be represented by a product of superoperators \( S_1, S_2, \ldots, S_n \) acting on the vectorized initial density matrix \( |\rho_0\rangle\rangle \),
\[
|\rho_0\rangle\rangle \rightarrow S_n S_{n-1} \cdots S_1 |\rho_0\rangle\rangle.
\]

Unitary operators have many nice properties such as:
\begin{itemize}
    \item \( U^\dagger U = U U^\dagger = I \) (unitarity),
    \item \( \lambda(U) = \{ e^{i\theta} : \theta \in \mathbb{R} \} \) (eigenvalues on the unit circle),
    \item \( \|Ux\|_2 = \|x\|_2 \) (norm preservation).
\end{itemize}

In contrast, superoperators for quantum channels must be \emph{completely positive and trace preserving} (CPTP). For a map \( \Phi : L(H) \rightarrow L(H) \), this means:
\begin{itemize}
    \item For any positive semidefinite operator \( X \), \( \Phi(X) \) is also positive semidefinite.
    \item For any Hilbert space \( H' \), \( \Phi \otimes I_{H'} \) is also positive.
    \item \( \mathrm{tr}(\Phi(X)) = \mathrm{tr}(X) \) (trace preserving).
\end{itemize}

Completely positive maps are always positive, and every CPTP map
represents a valid quantum channel. The matrix representation of a
superoperator can be reshaped into a positive semidefinite matrix,
guaranteeing complete positivity. Other properties, such as trace
preservation and Hermiticity preservation, arise from related
transformations.

Choosing a formalism and parameterization that respects these
properties is crucial for any PMM that aims to learn the underlying
noise model of a quantum computer.

A possible goal of this project is to develop a PMM that can learn (perhaps a
low-dimensional representation of) the underlying open quantum system
nature of a real quantum circuit on a real quantum computer. Using
this, we aim to demonstrate zero-noise extrapolation using the PMM
that is more physically constrained and potentially more accurate than
current state-of-the-art ZNE (Zero-noise extrapolation) techniques.


By learning the underlying noise model of each gate on a physical
quantum computer, we can potentially use this information to construct
circuits that are more robust to noise.


\subsection{Implementing Hamiltonians onto a Circuit}

Here one could use Qiskit to implement a Hamiltonian onto a circuit. As an examnple, we could start with a classic, the   Ising Model:
\[
H = B \sum_i X_i + J \sum_i Z_i Z_{i+1},
\]
where \( X \) and \( Z \) are Pauli operators. See Ref.~\cite{smith2019simulating} for guidance on encoding many-body Hamiltonians.

\subsection{Zero-Noise Extrapolation}

One could then
study sources of noise during computation and the presence of
different noise channels. A useful reference is Nielsen and Chuang’s
\emph{Quantum Computation and Quantum Information}. The depolarizing
noise channel is typically used in ZNE. Implement ZNE following
Ref.~\cite{giurgica2020digital}, or use the \texttt{mitiq} Python
library (\url{https://mitiq.readthedocs.io/}).

Note to self: need to perform a literature search on ZNE techniques enhanced
with neural networks (recent results within the last 1–2 years) to
possibly include as comparison.

\subsection{Exact Simulation of Simple Circuit}

One could then implement an exact unitary simulation of a simple
circuit, such as Bell-state generation and measurement. Then extend it
to an exact noisy simulation.

\subsection{PMM for Noiseless Simple Circuit}

The next step is to implement a PMM for the noiseless circuit and
train it to reproduce observables or states. Explore larger circuits
to study potential dimensionality reduction.

\subsection{PMM for Noisy Simple Circuit}


Develop a PMM for the noisy circuit. Demonstrate that it can learn the
underlying noise model and extrapolate to the noiseless case. Explore
larger circuits as before.

\subsection{Comparison with Traditional ZNE}

Compare PMM performance with traditional ZNE methods for the simple
circuit. This may involve real quantum hardware or realistic noise
simulators. Here one could
demonstrate the PMM method on real quantum hardware, comparing it with existing ZNE techniques such as those in \texttt{mitiq}. One could extend this to include tensor-network approaches (MPS, DMRG, TEBD) to reduce computational complexity.

\section*{References}

\begin{thebibliography}{9}

\bibitem{manzano2019}  D. Manzano, \emph{A Short Introduction to the Lindblad Master Equation}, \href{https://arxiv.org/abs/1906.04478}{arXiv:1906.04478}, 2020.
\bibitem{yuen2022}  H. Yuen, \emph{Lecture 2: Mixed States and Density Matrices}, \href{https://www.henryyuen.net/spring2022/lec2-mixed-states.pdf}{2022}.

\bibitem{giurgica2020digital}
T.~Giurgica-Tiron, Y.~Hindy, R.~LaRose, A.~Mari, and W.~J.~Zeng,
\newblock ``Digital zero noise extrapolation for quantum error mitigation,''
\newblock in \emph{IEEE International Conference on Quantum Computing and Engineering (QCE)}, pp. 306–316, 2020.

\bibitem{smith2019simulating}
A.~Smith, M.~S.~Kim, F.~Pollmann, and J.~Knolle,
\newblock ``Simulating quantum many-body dynamics on a current digital quantum computer,''
\newblock \emph{npj Quantum Information}, 5(1):106, 2019.

\bibitem{wood2015tensor}
C.~J.~Wood, J.~D.~Biamonte, and D.~G.~Cory,
\newblock ``Tensor networks and graphical calculus for open quantum systems,'' 2015.

\end{thebibliography}

\end{document}


